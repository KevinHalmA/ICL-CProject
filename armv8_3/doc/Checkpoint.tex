\documentclass{article}

\usepackage[english]{babel}

\usepackage[letterpaper,top=2cm,bottom=2cm,left=3cm,right=3cm,marginparwidth=1.75cm]{geometry}

\usepackage{amsmath}
\usepackage{graphicx}
\usepackage[colorlinks=true, allcolors=blue]{hyperref}

\date{June 9, 2023}

\title{C Interim Checkpoint Report}
\author{Group 3 \\ Andrei Braguta, Jiawen Li, Ruiqi Zhang, Kevin Halm Alvarez}
\begin{document}
\maketitle

\section{Introduction}

Our group has worked on developing a C-based emulator for executing ARMv8
AArch64 object code. This emulator operates in both 32-bit and 64-bit register
modes, involving 2MB of byte-addressable memory and 31 general-purpose
registers, alongside the zero register, program counter and processor state
register. Our emulator reads in a binary file of object code and executes the
main emulator pipeline by fetching, decoding and executing each instruction
until the halt instruction is reached. Upon termination, it writes to an output
file, detailing the final emulator state.

\section{Group Organisation}

\subsection{Work Distribution}

To ensure effective collaboration within the group, we utilised a combination of
pair programming and section delegation. Each of the four members was assigned a
particular instruction type: data processing (immediate), data processing
(register), single data transfer and branch. This allowed us to work
independently on our allocated instruction types (with an exception due to an
overlap between arithmetic operations in immediate and register instructions).
When designing the general structure of our emulator, we collaborated closely as
a team. Together, we defined and implemented data structures representing the
CPU state and the various instruction types. This collaborative approach allowed
us to make individual contributions while ensuring that we were all on the same
page regarding the main emulator structure, and on the whole allowed for a
shared understanding amongst all group members to make sure each member could
carry out their assigned tasks.

\subsection{Group Structure}

To collaborate well within our group, we adopted a number of strategies.
Firstly, we established a group chat where we could address any coding queries,
provide Git updates, schedule online or in-person meetings, and share relevant
files. This allowed for clear communication and streamlined our workflow. In
terms of branch control, we created personal branches in the GitLab repo. This
allowed each member to work on their respective implementations independently by
committing and pushing to their own branch. Regularly merging these branches
with the master branch helped us avoid conflicts and bugs in our code. This
approach enabled us to maintain a steady progress rate and ensured that anyone
facing difficulties or falling behind could quickly catch up with the rest of
the group. These strategies allowed us to complete the emulator in our predicted
time frame while also contributing equally and efficiently. Hence, we plan to
keep our group structure as is for the rest of the project.

\section{Emulator Implementation}

\subsection{Emulator Structure}

We have represented the emulator state using a struct named CPUState located in
utilities.h. CPUState contains a pointer to memory (dynamically allocated on the
heap), 31 general-purpose registers (stored in an array), the zero register, the
program counter and PState (a struct containing booleans representing the 4
condition flags). To represent the general structure of an instruction, we have
created a struct named Instr - this contains an enum representing the
instruction type and a union of structs representing the bit format of each
individual instruction type. 

The main entry point to our emulator program is via emulate.c, which handles the
loading of the binary file into memory (via functions defined in
binary\_loader.c), the initialisation and execution of the emulator, and the
writing of the emulator state to a given .out file (via functions defined in
emulator.c). emulator.c implements the execution pipeline of the emulator, with
a fetch function that loads the next instruction from memory, and the main
decode and execute functions, which delegate the actual decoding/execution of
the instruction to functions for that particular instruction type. These
specialised decode and execute functions are located in individual source files
for each instruction type (along with dp\_arithmetic.c, which handles the
arithmetic operations shared by immediate and register).

Finally, we have defined utility functions for shifting, sign extending, etc in
utilities.h and helper functions for reading/writing to memory and registers
(using the appropriate bit mode) in memory.c and registers.c respectively.

\subsection{Code Re-usability}

To streamline our implementation of the assembler in Part 2, we intend to reuse
several of the helper functions defined in utilities.c (eg: get\_bit\_mask), and
in particular the internal instruction representation that we defined for the
emulator in instructions.h. This will allow us to separate the assembly process
into first parsing the assembly code into our internal representation of the
instruction, and then converting it into binary object code, meaning that we can
maintain a modular design for the assembler. As a result, we may need to
refactor our src directory structure and Makefile, potentially by creating
separate directories for the assembler and emulator and a common directory
containing files shared between them.

\section{Future Parts}

\subsection{Challenges}

As a group, we find Part 3 to be the most daunting task due to its debugging
process. Constantly editing code, merging it, downloading it onto a USB, and
attempting to boot the Raspberry Pi could prove to be quite tedious. Moreover,
the fact that we only have one Raspberry Pi for testing means that debugging
activities may require the presence of all group members. Part 4 may also
present some challenges due to its open-ended nature; there is a possibility of
either choosing a brief that is too simple, or getting carried away and
attempting to accomplish something which may not be realistically achievable
within the given time frame.

\subsection{Implementation Strategies}

In order to ensure a smooth execution of Part 3, we have implemented several
initiatives. These include conducting bi-daily in-person debugging meetings,
either in labs or at the accommodations of our group members. Additionally, we
plan to create a dedicated branch specifically for this project section,
ensuring that the correct version of our code is always downloaded and uploaded
to our USB. We merge this branch to the master branch after verification.
Moreover, to handle the complexities of different sections effectively, we
intend to increase our utilisation of pair programming. Regarding Part 4, we are
currently in the process of identifying a project idea that strikes a balance
between difficulty and feasibility within the given time frame. Our focus is on
generating a creative concept while maintaining good code style and hygiene,
fully leveraging the capabilities of the C programming language.

\section{Conclusion}

Overall, we have approached the project with a combination of effective
planning, delegation, and interesting implementations within different sections
of the emulator. Our aim has been to showcase our proficiency in C programming
and demonstrate strong teamwork throughout the project.

\end{document}